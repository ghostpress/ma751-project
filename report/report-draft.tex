\documentclass{article}

\usepackage{amsmath}
\usepackage{amsfonts}
\usepackage[normalem]{ulem}
\usepackage{bm}
\usepackage{xcolor}
\usepackage{algorithm}
\usepackage{algpseudocode}
\usepackage{graphicx}
\usepackage{mathtools}
\usepackage{amsthm}

\newcommand\numberthis{\addtocounter{equation}{1}\tag{\theequation}}
\setlength\parindent{0pt}

\title{A Review of Methods for Regression of Glacier Evolution}
\author{Firas Abboud, Xuan Loc Huynh, Luc\'ia Vilallonga}
\date{April 30th, 2025}  

\newtheorem{theorem}{Theorem}[section]
\newtheorem{lemma}[theorem]{Lemma}

\begin{document}
\maketitle

%%%%%%%%%%%%%%%%%%%%%%%%%%%%%%%%%%%%%%%%%%%%%%%%%%%%%%%%%%%

\section{Outline (delete later)}

\begin{enumerate}
    \item Introduction/background on glaciology, glacier modeling (can pull from proposal and Firas' notes)
    \item Paper inspiration: Bobilar Deep Learning Applied to Glacier Evolution Modeling \cite{Bobilar:etal:2020}
    \item Adapting the methods: new dataset, different set of models
    \begin{enumerate}
        \item Data exploration
        \item Linear models mostly unsuccessful, data highly nonlinear and esp. climate data is very inter-correlated
        \item Motivates use of other methods, in particular decision trees and ML
    \end{enumerate}
    \item Results
    \item Discussion
    \begin{enumerate}
        \item Difficulty in applying PINNs to glaciology modeling (mostly lack of data) - cite Loc's paper
    \end{enumerate}
\end{enumerate}

%%%%%%%%%%%%%%%%%%%%%%%%%%%%%%%%%%%%%%%%%%%%%%%%%%%%%%%%%%%

\section{Introduction}

This paper presents a machine learning (ML) approach to modelling annual glacier-wide surface mass balance (SMB) 
as a parameterized alternative to typical linear methods. SMB, which describes the difference between a glacier’s 
mass accumulation in the winter and ablation in the summer, is a key process that drives glacier evolution on 
regional and sub-regional scales. As such it is also critical to understanding and predicting environmental 
challenges such as sea-level rise and the disruption of ocean currents and stable weather patterns globally \cite{earth:org}. \\

Bobilar et. al., 2020 find that a fully-connected feed-forward model with 6 hidden layers outperforms Ordinary 
Least-Squares (OLS) and Least Absolute Shrinkage and Selection Operator (LASSO), which are traditionally used in 
parameterized approaches to simulating SMB from data \cite{Bobilar:etal:2020}. However, ML does not provide the 
same interpretability that OLS and LASSO do, and the authors note that one challenge of the method is the limited 
data available - which makes it more likely that an overparameterized ML model will overfit. Furthermore, some key 
factors to predicting SMB may not be captured in the data, such as physical processes described by dynamical systems. \\

\textcolor{red}{Add more to this, motivate methods used below. Also may want to cite other glaciology papers.}

%%%%%%%%%%%%%%%%%%%%%%%%%%%%%%%%%%%%%%%%%%%%%%%%%%%%%%%%%%%

\section{Our Methods}

%%%%%%%%%%%%%%%%%%%%%%%%%%%%%%%%%%%%%%%%%%%%%%%%%%%%%%%%%%%

\section{Results}

%%%%%%%%%%%%%%%%%%%%%%%%%%%%%%%%%%%%%%%%%%%%%%%%%%%%%%%%%%%

\section{Discussion}

%%%%%%%%%%%%%%%%%%%%%%%%%%%%%%%%%%%%%%%%%%%%%%%%%%%%%%%%%%%

\pagebreak
\bibliographystyle{plain}
\bibliography{refs.bib}

\end{document}